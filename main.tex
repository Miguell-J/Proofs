\documentclass{article}
\usepackage[utf8]{inputenc}
\usepackage[T1]{fontenc}
\usepackage[brazil]{babel}
\usepackage{amsmath,amssymb,amsthm}
\usepackage{tikz-cd}

\title{Caderno de Provas Matemáticas}


\newtheorem{definition}{Definição}
\newtheorem{theorem}{Teorema}[section]
\newtheorem{corollary}{Corolário}[theorem]
\newtheorem{lemma}[theorem]{Lema}

\renewcommand\qedsymbol{$\blacksquare$}

\begin{document}

\maketitle

\tableofcontents
\newpage

\section{Fáceis-Médios-Dificeis}

\subsection{Prove que a soma de dois pares também é par}

\begin{definition}
    \label{par}
    Um número inteiro $x \in \mathbb{Z}$ é par se existe $k \in \mathbb{Z}$ tal que $x = 2k$.
\end{definition}

\begin{theorem}[Teorema da soma de pares]
\label{theo:1}
Dado 2 números pares $a, b \in \mathbb{Z}$, a soma de $a + b$ também é par.
\end{theorem}

\begin{proof}
Sejam $a, b \in \mathbb{Z}$ tal que $a,b$ sejam pares. Então existem $k,m \in \mathbb{Z}$ tais que:

$$ a = 2k, b = 2m$$

Somando as duas expressões

$$
    a + b = 2k + 2m = 2(k + m)
$$

Como $k, m \in \mathbb{Z}$, temos que $a+b = 2n$, sendo $n = k + m \in \mathbb{Z}$. Logo $a+b$ é par.
\end{proof}




\subsection{Prove que a raiz quadrada de 2 é irracional.}

\begin{definition} 
    Um número $x \in \mathbb{Q}$ é racional se existe $x = \frac{a}{b}$, aonde $a,b \in \mathbb{Z}$, $b \neq 0$ e $gcd(a, b) = 1$
\end{definition}

\begin{theorem}
    [Teorema da irracionalidade de $\sqrt{2}$]
    \label{theo:2}
    Dado $n = \sqrt{2}$, $n \notin \mathbb{Q}$
\end{theorem}

\begin{lemma}
\label{lemma:Lema1}
Se o quadrado de um número é par (veja \ref{par}), então o número também é par.
\end{lemma} 

\begin{proof}
    Consideremos $\sqrt{2} \in \mathbb{Q}$, logo $\sqrt{2} = \frac{a}{b}$ com $a, b \in \mathbb{Z}$, $b\neq 0$ e $gcd(a, b) = 1$

    Se elevarmos ao quadrado:

    $$2 = \frac{a^2}{b^2}$$

    O que nos leva a:

    $$2b^2 = a^2$$

    Como $a,b \in \mathbb{Z}$ e $a^2 = 2b^2 $, então $a$ é par (veja \ref{par}). Portanto podemos substituir $a$ como:

    $2b^2 = 4k^2$

    Se dividirmos ambos os lados por 2:

    $$b^2 = 2k^2$$

    O que nos leva a descobrir que $b$ também é par. E dado que dois números pares $a,b$ não tem $gcd(a,b) = 1$, isso nos leva a uma contradição, provando que $\sqrt{2} \notin \mathbb{Q}$

\end{proof}

\subsection{Seja $f: A \to B$. Prove que se $f$ é injetora, então $f^{-1}(f(C)) = C$ para todo $C \subseteq A$}

\begin{definition}
    Uma função \( f: A \to B \) é dita \textbf{injetora} se:
    \[
        \forall x_1, x_2 \in A, \quad f(x_1) = f(x_2) \Rightarrow x_1 = x_2.
    \]
    Ou seja, diferentes elementos no domínio têm imagens diferentes.
\end{definition}

\begin{definition}
    Dada uma função \( f: A \to B \), definimos a função inversa à esquerda \( f^{-1}: \operatorname{Im}(f) \to A \), tal que:
    \[
        \forall x \in A, \quad f^{-1}(f(x)) = x.
    \]
    Essa inversa é bem definida se \( f \) for injetora.
\end{definition}

\begin{lemma} [Reconstrução de valor via função injetora inversa]
    \label{func inj}
    Seja \( f: A \to B \) uma função injetora. Então, para todo \( x \in A \),
    \[
    f^{-1}(f(x)) = x.
    \]
\end{lemma}

\begin{theorem}
Seja \( f: A \to B \) uma função injetora. Então, para todo subconjunto \( C \subseteq A \),
\[
    f^{-1}(f(C)) = C.
\]
\end{theorem}

\begin{proof}
Seja \( C \subseteq A \). Vamos provar as duas inclusões:

\textbf{1. \( f^{-1}(f(C)) \subseteq C \):}

Seja \( x \in f^{-1}(f(C)) \). Então \( f(x) \in f(C) \), ou seja, existe \( c \in C \) tal que \( f(x) = f(c) \).

Como \( f \) é injetora, temos \( x = c \), então \( x \in C \). Logo, \( f^{-1}(f(C)) \subseteq C \).

\textbf{2. \( C \subseteq f^{-1}(f(C)) \):}

Seja \( x \in C \). Então \( f(x) \in f(C) \), e assim \( x \in f^{-1}(f(C)) \).

Portanto as duas inclusões mostram que \( f^{-1}(f(C)) = C \).
\end{proof}

\subsection{Prove que se $A \subseteq B$, então $\mathcal{P}(A) \subseteq \mathcal{P}(B)$ (Onde $\mathcal{P}(X)$ é o conjunto das partes de $X$)}

\begin{definition}
    Seja $X$ um conjunto, o conjunto das partes $\mathcal{P}(X)$ é o conjunto de todos os subconjuntos de $X$. Todo conjunto das partes inlui o conjunto vazio $\emptyset$, e todos os subconjuntos de $X$, incluindo o próprio $X$.
    
\end{definition}

\begin{lemma}
    \label{Conjunto das partes}
    Seja $X$ um conjunto e $\mathcal{P}(X)$ seu conjunto das partes, se $X$ possui $n$ elementos, então $\mathcal{P}(X)$ possuirá $2^n$ elementos. 
\end{lemma}

\begin{theorem}
    Sejam dois conjuntos $A$ e $B$, dado $A \subseteq B$, temos que $\mathcal{P}(A) \subseteq \mathcal{P}(B)$ 
\end{theorem}

\begin{proof}
    Sejam $A$ e $B$ dois conjuntos, sendo que $A \subseteq B$ e seus conjuntos das partes $\mathcal{P}(A), \mathcal{P}(B)$. Portanto temos que:

    $$\forall x \in A \subseteq B$$

    Dessa forma, (veja \ref{Conjunto das partes}) $\mathcal{P}(A) \subseteq \mathcal{P}(B)$. Ou seja,  todo subconjunto possível de $A$ também está em $B$. Assim podemos concluir que $\mathcal{P}(A) \subseteq \mathcal{P}(B)$.
\end{proof}

\subsection{Prove que se $n^2$ é par, então $n$ é par.}

\begin{definition}
    Um número inteiro $n \in \mathbb{Z}$ é par se existe $k \in \mathbb{Z}$ tal que $n = 2k$.
\end{definition}

\begin{theorem}
    Seja $n \in \mathbb{Z}$. Se $n^2$ é par, então $n$ também é par.
\end{theorem}

\begin{proof}
    Iremos provar pela contrapositiva se $n$ é ímpar, então $n^2$ é ímpar. Se $n$ é ímpar, então existe $k \in \mathbb{Z}$ tal que $n = 2k + 1$. Calculando o quadrado:

    $$
    n^2 = (2k + 1)^2 = 4k^2 + 4k + 1 = 2(2k^2 + 2k) + 1
    $$

    Portanto, $n^2$ é ímpar. Logo, a contrapositiva é verdadeira, e isso implica que se $n^2$ é par, então $n$ também é par.
\end{proof}

\subsection{Prove que o limite $\lim_{x \to 2} (3x + 1) = 7$ usando a definição $\epsilon-\delta$.}

\begin{definition}
    Seja $f: \mathbb{R} \to \mathbb{R}$ e $a, L \in \mathbb{R}$. Dizemos que
\[
\lim_{x \to a} f(x) = L
\]
se, para todo $\varepsilon > 0$, existe um $\delta > 0$ tal que, sempre que
\[
0 < |x - a| < \delta, \quad \text{então} \quad |f(x) - L| < \varepsilon.
\]
\end{definition}

\begin{proof}
    Seja o limite $\lim_{x \to 2} (3x + 1) = 7$, podemos analisar a veracidade do limite. Queremos mostrar que para todo $\varepsilon > 0$, existirá um $\delta > 0$ tal que:

    $$0 < |x - 2| < \delta, \quad \text{e} \quad |f(x) - 7| < \varepsilon$$

    Dessa forma temos:

    $$\quad |3x+1 - 7| < \varepsilon = \quad |3x - 6| < \varepsilon = \quad 3|x - 2| < \varepsilon = \quad |x - 2| < \frac{\varepsilon}{3}$$

    Assim podemos escolher $\delta = \frac{\varepsilon}{3}$. Portanto, com $0 < |x - 2| < \frac{\varepsilon}{3}$ temos o limite verificado, e a prova está completa.

    

    
\end{proof}

\subsection{Prove que uma função constante é contínua em todos os pontos. (Use definição de continuidade: $\forall \varepsilon \gt 0, \exists \delta \gt 0...$)}

\subsection{Prove que existe infinitos números primos.}

\subsection{Prove que $\sqrt{3} \notin \mathbb{Q}$ mas $\sqrt{9} \in \mathbb{Q}$}

\subsection{Prove que $\mathbb{Q}$ é denso em $\mathbb{R}$ (Para todo $x, y \in \mathbb{R}$ com $x \lt y$ existe $q \in \mathbb{Q}$ tal que $x \lt q \lt y$)}

\subsection{Mostre que a função $f(x) = x^2$ não é uniformemente contínua em $\mathbb{R}$}

\subsection{Prove que a união de duas aberturas em $\mathbb{R}$ é um conjunto aberto.}

\subsection{Prove que todo subconjunto finito de $\mathbb{R}$ é fechado.}

\subsection{Prove que não existem inteiros $x, y \in \mathbb{Z}$ tais que $x^3 + y^3 = 2024$}

\subsection{Seja $G$ um grupo com a operação $*$ e $a \in G$ tal que $a^2 = e$ (elemento neutro). Prove que $a = a^{-1}$}

\subsection{prove que se $f: \mathbb{R} \to \mathbb{R}$ é derivavel em $a \in \mathbb{R}$ então ela é contínua em $a$}

\subsection{prove que a sequência $a_n = \frac{1}{n}$ converge e que seu limite é 0.}

\subsection{Prove que toda sequência monótona crescente e limitada superiormente é convergente.}

\subsection{Determine se a série $\sum_{n=1}^{\infty} \frac{1}{n^2}$ converge}

\subsection{Prove que se $\sum a_n$ converge absolutamente, então também converge.}

\subsection{Prove que um conjunto de vetores ortogonais não nulos em $\mathbb{R}^n$ é linearmente independente. (Use definição de ortogonalidade e combinação linear)}

\subsection{Seja $T: \mathbb{R}^n \to \mathbb{R}^n$ uma transformação linear tal que $T^2 = T$. Prove que $\operatorname{Im}(T) \cap \ker(T) = \{0\}$}

\subsection{Mostre que o conjunto $\mathbb{Q} \subset \mathbb{R}$ não é fechado.}

\subsection{Mostre que o intervalo $(0,1) \subset \mathbb{R}$ não é compacto.}

\subsection{Prove que todo operador unitário $U$ em um espaço de Hilbert de dimensão finita preserva normas, ou seja, $\|Ux\| = \|x\|$ para todo $x$.}

\subsection{Mostre que o conjunto dos estados quânticos (vetores normalizados em $\mathbb{C}^n$) forma uma variedade topológica esférica complexa. (dica, $S^{2n-1} \subset \mathbb{C}^n$ com norma 1)}

\subsection{Mostre que a classe $P$ é fechada sob composição de funções.}

\subsection{Mostre que se $\mathbf{NP} = \mathbf{coNP}$ então o complemento de qualquer problema NP-completo também é NP-completo.}

\subsection{Prove que todo grupo abeliano finito é isomorfo a um produto de grupos cíclicos de ordem potência de primos.}

\subsection{Mostre que o primeiro grupo de homologia $H_1(S^1) \cong \mathbb{Z}$}

\subsection{Prove que a função grau de uma aplicação contínua $f: S^1 \to S^1$ é um invariante topológico sob homotopia. (Ou seja, se $f \sim g$ então $\deg(f) = \deg(g)$)}

\subsection{Mostre que a functorialidade da homologia garante que mapas homotópicos induzem o mesmo homomorfismo em homologia.}

\subsection{Mostre que uma superfície toroidal tem grupo de homologia $H_1(T^2) \cong \mathbb{Z} \oplus \mathbb{Z}$}

\subsection{Mostre que qualquer operador linear hermitiano em $\mathbb{C}^n$ possui uma base ortonormal de autovetores e autovalores reais.}

\section{Insano}

\subsection{Prove ou discuta criticamente a equivalência entre a hipótese de Riemann e a estimativa de erro na contagem de primos: $\pi(x) = \text{Li}(x) + O\left(x^{1/2} \log x\right)$ (Assuma a validade ou estude implicações formais. Construa uma prova da implicação RH ⟹ estimativa.)}

\subsection{Prove que o grupo fundamental do toro $T^2$ é $\mathbb{Z} \oplus \mathbb{Z}$  e mostre como ele classifica os loops essenciais da superfície. (Generalize a estrutura para superfícies orientáveis genus $g$)}

\subsection{Mostre que a categoria dos espaços de Hilbert finitamente dimensionais com operadores unitários é uma categoria monoidal fechada, e discuta implicações para circuitos quânticos.
(Use a linguagem de funtores, objetos internos e tensores)}

\subsection{Estude o problema de Hodge em $\mathbb{C}P^2$. Verifique, com provas formais, que todo ciclo harmônico em $H^{2}( \mathbb{C}P^2 )$ é classe de um subvariedade algébrica.
(Esse é um caso conhecido da conjectura de Hodge)}

\subsection{Mostre que toda variedade diferenciável compacta orientável de dimensão 3 admite uma estrutura de fibrado de Seifert. (Esse é um caso particular da geometrização de Thurston que implica a Poincaré)}

\subsection{Assuma que $\mathbf{P} = \mathbf{NP}$, e demonstre que então o problema de determinar se um grafo tem um ciclo Hamiltoniano pode ser resolvido em tempo polinomial. (Formalize usando reduções e classes de complexidade)}

\subsection{Prove que o espectro de um operador compacto autoadjunto em um espaço de Hilbert é um conjunto enumerável com possível ponto de acumulação apenas em 0. (Fundamental para física quântica e teoria espectral)}

\subsection{Mostre que os grupos de homotopia estáveis das esferas são finitos para $n \gt 0$  exceto para múltiplos especiais. (Use ideias da teoria de espectros e suspensões estáveis)}

\subsection{Mostre que todo topos booleano com um objeto natural e um subobjeto de igualdades satisfaz os axiomas de Peano internamente. (Traduza isso em lógica categórica e modelagem interna)}

\subsection{Formalize como a homologia persistente pode ser usada para extrair topologia de espaço-tempo em abordagens tipo causal set theory. (Trabalho de fronteira entre física matemática, topologia e IA)}



\end{document}